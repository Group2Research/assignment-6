\documentclass[11pt]{article}
\usepackage{zed-csp,graphicx,color}%from
\begin{document}
\begin{titlepage}

\centerline{COLLEGE OF COMPUTING AND INFORMATIC SCIENCES}
\paragraph{•}
\centerline{DEPARTMENT OF COMPUTER SCIENCE\\}
\paragraph{•}

\centerline{COURSEWORK: RESEARCH METHODOLOGY(BIT 2207)\\}
\paragraph{•}

\centerline{LECTURER: MR.ERNEST MWEBAZE}


\paragraph{•}
\centerline{COMPILED BY:
 GROUP 2}
 \paragraph{•}
\centerline{KYOBE JEREMIAH   16/U/6485/EVE    216013670}
\centerline{WAFULA DERRICK    16/U/20275/EVE    216021715}
\centerline{MUKIIBI PAUL       16/U/7479/EVE     216014896}
\centerline{OLUMA  RODERICK   16/U/10985/EVE    216012431}

\paragraph{•}
\end{titlepage}
\pagenumbering{roman}
\tableofcontents
\newpage
\pagenumbering{arabic}
\section{TOPIC}
AN ONLINE SALON LOCATING MOBILE APPLICATION WORKING WITHIN MAKERERE SUB-COUNTY.

\section{INTRODUCTION}
1.0	 Overview\\
Salons are everywhere, and if you're planning to get into the salon business, it's important to understand\cite{price1999commercial} how you can differentiate yourself and attract clients. Want to boost sales and increase productivity by more than (40 percent)? Learn how to use locator mobile applications to automatically increase ease of access to your fine salon and services provided.A salon locator mobile application may seem to be chore but in 2018, it's a strategic way to bit your competitors. This proposal includes my methods for gathering information and a schedule for completing the review.\cite{dick2005systematic}
 

\section{PROBLEM STATEMENT}There is a tremendous need for coordinated and accessible home-based services for the youths in Makerere sub-county. According to the Makerere sub-county Survey of the innovative campusers completed by Makerere IT innovation department in June of 2016, the western areas \cite{chiwanga2016urban}of the sub-county have some of the highest populations of occupants in the sub-county with the highest in Kikoni (21 percent). Elder Help has identified additional census tracks within the communities of Wandegeya, Kavule and Kubiri where (35 percent)of the population are campusers from the ages of 18 to 30. According to the survey, the need to look trendy by getting the latest haircuts and styles easily is unbelievable, making the salon business a boom. Many are females who anxiously desire to look better and classy than ever before. As the economy continues to decline and the state is getting more technological, the need for more cost effective and coordinated community-based solutions to ease income flow in for salon business owners throughout the areas is on the rise.\cite{makina2004impact}

\section{AIMS / OBJECTIVES}
•	To ensure ease of access and location of quality salons around Makerere sub county
•	To help salon business owners in Makerere sub county differentiate themselves and hence attract more clients
•	To improve on the nature growth of the salon business in Makerere sub county
•   To enable salon business owners boost sales and increase productivity\cite{stockdale2012identifying}
 
\section{SCOPE}
The investigation was carried out around Makerere University covering all faculties of the University. At least one student from each faculty was handed a questionnaire which they filled and handed back to me. 
\section{METHODOLOGY}
The research was carried out in majorly two stages i.e data collection methods and data Processing and Analysis.
\subsection{Research Design}
The population of this study was 100 residents of Makerere sub county from the different social groups around the locality.These were randomly selected to answer questionnaires about qualities of a standard salon,service price range,accepted situation of a salon in an area. The questionnaires contained structured questions composed of the business performance,its influence on the community,and what can be done to\cite{stockdale2012identifying} better it.
\subsection{Population Size}
The population of this study is 100 residents of Makerere sub county from different groups of people especially youths of the area and salon business owners who were randomly selected to answer questionnaires about the nature of the salon business in the area. 
\subsection{Sampling Frame}
The investigation was carried out around Makerere sub county covering all the villages in the area. At least ten youths and five salon business owners from each village were handed a questionnaire which they filled and handed back to me.
\subsection{Research Procedure}
\subsection{Desk Study}
This study mainly considered reports from experienced salon business owners and the various youth representatives.

\subsection{Data collection methods}
This stage will work out the secondary data from the study of the research on salon business and quality standards\cite{sachdev2004relative} capable of fulfilling the conditions mentioned in the problem statement and aim of the study.
The available data would be reviewed and assured that it covers all essential criteria of salon business enhancement.Also informal interviews and discussions will be conducted with experienced salon business owners to endorse the reliability of the available data and quality issues of the business.At this stage,it is suggested that salon business quality data published by lead research will be used for development of salon locator mobile application.The study has been conducted to determine the quality of service quality indicators and parameters for affordable salons in Makerere sub county. 
   The study identified that absence of quality salons in Makerere resulted from various factors ranging from absence of laws concerning salon standards ,policy failure, inadequate capital for salon business investors to provide standard services , low levels of professional requirements from the salon workers and the unwillingness of the salon users spend enough on salon services.
The study concludes that existing level quick access or location of affordable and yet quality salon services in Makerere could be enhanced through improving the design and technique of service delivery, and sustainability of customers by going more technological for example embracing the use of salon locator.

\subsection{Data Processing and Analysis}
At this stage, available data will be reviewed in context of quality indicators for salon services and classified under various segments of salon service quality assessment.This practice will identify the indicators to be used as a tool evaluate the salon service quality and will be correlated in context of quality indicators.The scale will be set to measure the response for each quality indicator.At this stage, the researcher will work in close collaboration with the software developers.\cite{cai2015challenges}

 \section{LITERATURE REVIEW}
\subsubsection{Introduction}
 This part will discuss the supporting literature on an ONLINE SALON LOCATING MOBILE APPLICATION process which in this study is also referred to as the information system development process. The implementation of new information systems is a significant investment for organisations. The organisations in the world have come to the conclusion that they cannot rely on human interventions alone because the lack of technological advancement will threaten the existence of any organization. Due to the intervention of computerised applications technology has become the ultimate tool to replace hard copy processes
The traditional system of hair salon was manual and insecure because there was no any counting system of\cite{kuh1993their}
customers coming in the salon which creates sometimes major issues. The customer as well as the owner  faces the problems.

\subsection*{Technological concepts}
These early systems are dependent on paper pencil systems for billing purpose that means the records
of bills of customer and the workers working in their salon are in written form. The records may get wrong due to anyone's mistake. There is difficulty in maintaining records of all these tasks manually. Hence, proposed system is the best solution of avoiding all these problems.Salon manager are often responsible scheduling staff members, training new front desk workers.
Since salon is a service sector, so the success of hair salon depends on the satisfaction of customer. The business of salon is totally dependent on customer satisfaction.

The term location-based services (LBS) is a recent concept that denotes applications integrating geographic location (i.e., spatial coordinates) with the general notion of services. Examples of such applications include emergency services, car navigation systems, tourist tour planning, or "yellow maps" (combination of yellow pages and maps) information delivery.
With the development of mobile communication, these applications represent a novel challenge both conceptually and technically. Clearly, most such applications will be part of everyday life tomorrow, running on computers, personal digital assistants (PDAs), phones, and so on. Providing users with added value to mere location information is a complex task. Given the variety of possible applications, the basic requirements of LBS are numerous. Among them we can cite the existence of standards, efficient computing power, and friendly yet powerful human–computer interfaces.(Jochen Schiller, Freie Universität Berlinr,).

The mobile phones have revolutionized the communication and drastically affected to the life style of the modern nomadic people.  The voice capabilities of the mobile phones are currently augmented with data capabilities of increasing speed. The small size mobile terminals  mobile phones and PDAs are converging and evolving into Personal  Trusted  Devices  (PTD),  which  allows  users  to  access  Mobile  Internet services  and  run  applications  at  any  time  and  at  any  place.  The telecommunication 
industry  estimates  that  by  2003  there  will  be  about  500  million  Internet enabled mobile terminals in the world. The number of these mobile Internet enabled terminals is  expected  to  exceed  the  number  of  fixed  line  Internet  users  around  2003

The rapidly growing population of PTD users generates huge markets for related services, offering new attempting opportunities for business. The inherent features of PTDs are their high portability and personal nature. They are used  for  storing  and  accessing  information  at  any  time  wherever  the  users  go.  The continuous  availability  of  the  device  and  the  emerging  capability  of  the  terminals and/or the mobile network infrastructure to position the terminals on the earth allows new  types  of  spatiotemporal  real time  services  that  are  called  Location Based Services (LBS). LBSs are services accessible with PTDs through the mobile network and utilizing the ability to make use of the location of the terminals. Major part of the future Mobile Internet services is expected to be LBSs. The development of LBSs for mobile terminals got a strong impetus when US Federal Communications   Commission   (FCC)   set   the   Wireless   E911   Rules,   initially   in September 1999, requiring that it should be possible to locate all of the mobile phones for emergency purposes with the accuracy of about 100 meters in 67 percent of the cases.(Kirsi  Virrantaus,).\cite{hengshan6finding}

Since the dawn of the Web, online shoppers have mainly experienced electronic commerce through personal computers connected to the Internet via some form of fixed line. In the near future this may change, as many e-commerce transactions are expected to occur via a wide assortment of wireless and handheld devices (Economist Intelligence Unit, October 15, 2001). Wireless e-commerce is more commonly known as mobile or m-commerce, and, as noted in other chapters in this volume, is expected to develop into a significant market opportunity in the coming years throughout the world. Mobile operators in particular view m-commerce as a critical means of increasing average revenue per user (ARPU), since increasing competition has driven down prices for voice services at the same time that costs related to the transition to the next generation digital wireless infrastructure have risen.(Charles Steinfield,).



\subsection*{State of practice}

Salon Locator App has the following features of its client Management System;
There are daily uncertainities on how the day's incomes flow in will be due to the fluctuating customers at the place of work.
There are long queues while waiting for the customers that have been found at the salon  which wastes time.
Employee and customer details are stored in files.


\subsection*{State of the art}

This shows features of computerized daily updates for clients have been studied for reference.
 
Presence of  user guiding icons, Drop down menus but without entering full client details via the client portal for the side of salon owners.

\subsection*{Comparative evaluation}
We have gathered necessary information from various major salon businesses  and articles in regard to the features of their client Management Systems. We have also compared and contrasted their features; for which we have used their weaknesses as a stepping stone to our strength thus making our system to have additional features.

\section {METHODOLOGY}

\subsection*{Introduction}
This chapter encompasses the tools and techniques that were used to accomplish each of the specific objectives

\subsection*{System Study}
A detailed study to determine whether, to what extent, and how automatic data-processing equipment should be used; it usually includes an analysis of the existing system and the design of the new system, including the development of system specifications which provide a basis for the selection of equipment. 
To accomplish the systems study, the following tools and techniques were used;

\subsection*{Observation} 
Participatory observation was used as a means of finding out how the client management information system functions and its general set up. Observation as a fact finding method is unobtrusive, that is the users will not change their behavior or alter the facts because by watching their operations and therefore the real facts about the current system were noted.

\subsection*{Interviews}
This method was used to acquire information from the users majorly. Interviews were used because the Analyst can probe in great depth about the business’s work in respect to their client Management Information System which may not be achieved by other methods in a) above. Again personal contacts allow the analyst to be responsive and adaptive to what the user of the system to be designed says. And lastly a lot of time will be saved if the respondents are cooperative, highly responsive and brief to the point.


\subsection*{System Analysis}
System analysis is the process of collecting factual data, understand the processes involved, identifying problems and recommending feasible suggestions for improving the system functioning.
According to the Merriam-Webster dictionary, systems analysis is "the process of studying a procedure or business in order to identify its goals and purposes and create systems and procedures that will achieve them in an efficient way".
The tools that were used in system study are;

\subsection*{Flow charts.}
These are the pictorial representations of data. Flowcharts are used in designing and documenting complex processes or programs. Like other types of diagrams, they help visualize what is going on and thereby help the people to understand a process, and perhaps also find flaws, bottlenecks, and other less-obvious features within it.
Flow charts aid in;
•	Communication: - Flowcharts are better way of communicating the logic of a system to all concerned. 
•	Effective analysis: - With the help of flowchart, problem can be analysed in more effective way. 

•	Proper documentation: - Program flowcharts serve as a good program documentation, which is needed for various purposes.
On the other hand,flowcharts are disadvantageous because;
•	Complex logic: - Sometimes, the program logic is quite complicated. In that case, flowchart becomes complex and clumsy.
•	Alterations and Modifications: - If alterations are required the flowchart may require re-drawing completely.

\subsection*{Decision trees.} 
A decision tree is a decision support tool that uses a tree-like graph or model of decisions and their possible consequences, including chance event outcomes, resource costs, and utility. It is one way to display an algorithm.
Decision trees are commonly used in operations research, specifically in decision analysis, to help identify a strategy most likely to reach a goal.

\subsection*{System Design.}
\begin{itemize}
  \item Entity-relationship diagram
The entity-relationship diagram (ERD) allowed  us to describe the data involved in a real-world enterprise in terms of objects and their relationships and is widely used to develop an initial database design.
The foremost and most important ERD benefit is that it provides a visual representation of the design. It is normally crucial to have an ERD if you are looking to come up with an effective database design. This is because the patterns assist the designer in focusing on the way the database will primarily work with all the data flows and interactions. It is common to the ERD being used together with data flow diagrams so as to attain a better visual representation, an ERD clearly communicates the key entities in a certain database and their relationship with each other,more to that, an ERD model is quite flexible to use as other relationships can be derived easily from the already existing ones. 
 They are limited to;
•	No industry standard for notation: There is no industry standard notation \cite{kiers2000towards}     
•	Popular for high-level design: The E-R data model is especially popular for high level.

  \item\textbf{•} Data flow diagrams.
The Data Flow Diagram (DFD) is a graphical representation of the flow of data through an information system. It enables you to represent the processes in your information system from the viewpoint of data. The DFD lets you visualize how the system operates, what the system accomplishes and how it will be implemented, when it is refined with further specification.
Data flow diagrams are used by systems analysts to design information-processing systems but also as a way to model whole organizations. You build a DFD at the very beginning of your business process modeling in order to model the functions your system has to carry out and the interaction between those functions together with focusing on data exchanges between processes. You can associate data with conceptual, logical, and physical data models and object-oriented models.
There are two types of DFDs, both of which support a top-down approach to systems analysis, whereby analysts begin by developing a general understanding of the system and gradually break components out into greater detail:
•	Logical data flow diagrams - are implementation-independent and describe the system, rather than how activities are accomplished.
•	Physical data flow diagrams - are implementation-dependent and describe the actual entities (devices, department, people, etc.) involved in the current system.
\end{itemize}


\section{SYSTEM TESTING AND VALIDATION}
\subsection{Introduction}
This section shows the outcome of the software in the real work environment.
System testing of software or hardware is testing conducted on a complete, integrated system to evaluate the system's compliance with its specified requirements.
Computer system validation (CSV) is the documented process of assuring that a computer system does exactly what it is designed to do in a consistent and reproducible manner.
System testing is testing conducted on a complete, integrated system to evaluate the system compliance with its specified requirements. The following are the interfaces the user interacts with in the process of using this system; \cite{polson1992cognitive}



\bibliographystyle{IEEEtran}
\bibliography{ref}

\end{document}